% Options for packages loaded elsewhere
\PassOptionsToPackage{unicode}{hyperref}
\PassOptionsToPackage{hyphens}{url}
%
\documentclass[
]{article}
\usepackage{amsmath,amssymb}
\usepackage{lmodern}
\usepackage{iftex}
\ifPDFTeX
  \usepackage[T1]{fontenc}
  \usepackage[utf8]{inputenc}
  \usepackage{textcomp} % provide euro and other symbols
\else % if luatex or xetex
  \usepackage{unicode-math}
  \defaultfontfeatures{Scale=MatchLowercase}
  \defaultfontfeatures[\rmfamily]{Ligatures=TeX,Scale=1}
\fi
% Use upquote if available, for straight quotes in verbatim environments
\IfFileExists{upquote.sty}{\usepackage{upquote}}{}
\IfFileExists{microtype.sty}{% use microtype if available
  \usepackage[]{microtype}
  \UseMicrotypeSet[protrusion]{basicmath} % disable protrusion for tt fonts
}{}
\makeatletter
\@ifundefined{KOMAClassName}{% if non-KOMA class
  \IfFileExists{parskip.sty}{%
    \usepackage{parskip}
  }{% else
    \setlength{\parindent}{0pt}
    \setlength{\parskip}{6pt plus 2pt minus 1pt}}
}{% if KOMA class
  \KOMAoptions{parskip=half}}
\makeatother
\usepackage{xcolor}
\usepackage[margin=1in]{geometry}
\usepackage{color}
\usepackage{fancyvrb}
\newcommand{\VerbBar}{|}
\newcommand{\VERB}{\Verb[commandchars=\\\{\}]}
\DefineVerbatimEnvironment{Highlighting}{Verbatim}{commandchars=\\\{\}}
% Add ',fontsize=\small' for more characters per line
\usepackage{framed}
\definecolor{shadecolor}{RGB}{248,248,248}
\newenvironment{Shaded}{\begin{snugshade}}{\end{snugshade}}
\newcommand{\AlertTok}[1]{\textcolor[rgb]{0.94,0.16,0.16}{#1}}
\newcommand{\AnnotationTok}[1]{\textcolor[rgb]{0.56,0.35,0.01}{\textbf{\textit{#1}}}}
\newcommand{\AttributeTok}[1]{\textcolor[rgb]{0.77,0.63,0.00}{#1}}
\newcommand{\BaseNTok}[1]{\textcolor[rgb]{0.00,0.00,0.81}{#1}}
\newcommand{\BuiltInTok}[1]{#1}
\newcommand{\CharTok}[1]{\textcolor[rgb]{0.31,0.60,0.02}{#1}}
\newcommand{\CommentTok}[1]{\textcolor[rgb]{0.56,0.35,0.01}{\textit{#1}}}
\newcommand{\CommentVarTok}[1]{\textcolor[rgb]{0.56,0.35,0.01}{\textbf{\textit{#1}}}}
\newcommand{\ConstantTok}[1]{\textcolor[rgb]{0.00,0.00,0.00}{#1}}
\newcommand{\ControlFlowTok}[1]{\textcolor[rgb]{0.13,0.29,0.53}{\textbf{#1}}}
\newcommand{\DataTypeTok}[1]{\textcolor[rgb]{0.13,0.29,0.53}{#1}}
\newcommand{\DecValTok}[1]{\textcolor[rgb]{0.00,0.00,0.81}{#1}}
\newcommand{\DocumentationTok}[1]{\textcolor[rgb]{0.56,0.35,0.01}{\textbf{\textit{#1}}}}
\newcommand{\ErrorTok}[1]{\textcolor[rgb]{0.64,0.00,0.00}{\textbf{#1}}}
\newcommand{\ExtensionTok}[1]{#1}
\newcommand{\FloatTok}[1]{\textcolor[rgb]{0.00,0.00,0.81}{#1}}
\newcommand{\FunctionTok}[1]{\textcolor[rgb]{0.00,0.00,0.00}{#1}}
\newcommand{\ImportTok}[1]{#1}
\newcommand{\InformationTok}[1]{\textcolor[rgb]{0.56,0.35,0.01}{\textbf{\textit{#1}}}}
\newcommand{\KeywordTok}[1]{\textcolor[rgb]{0.13,0.29,0.53}{\textbf{#1}}}
\newcommand{\NormalTok}[1]{#1}
\newcommand{\OperatorTok}[1]{\textcolor[rgb]{0.81,0.36,0.00}{\textbf{#1}}}
\newcommand{\OtherTok}[1]{\textcolor[rgb]{0.56,0.35,0.01}{#1}}
\newcommand{\PreprocessorTok}[1]{\textcolor[rgb]{0.56,0.35,0.01}{\textit{#1}}}
\newcommand{\RegionMarkerTok}[1]{#1}
\newcommand{\SpecialCharTok}[1]{\textcolor[rgb]{0.00,0.00,0.00}{#1}}
\newcommand{\SpecialStringTok}[1]{\textcolor[rgb]{0.31,0.60,0.02}{#1}}
\newcommand{\StringTok}[1]{\textcolor[rgb]{0.31,0.60,0.02}{#1}}
\newcommand{\VariableTok}[1]{\textcolor[rgb]{0.00,0.00,0.00}{#1}}
\newcommand{\VerbatimStringTok}[1]{\textcolor[rgb]{0.31,0.60,0.02}{#1}}
\newcommand{\WarningTok}[1]{\textcolor[rgb]{0.56,0.35,0.01}{\textbf{\textit{#1}}}}
\usepackage{longtable,booktabs,array}
\usepackage{calc} % for calculating minipage widths
% Correct order of tables after \paragraph or \subparagraph
\usepackage{etoolbox}
\makeatletter
\patchcmd\longtable{\par}{\if@noskipsec\mbox{}\fi\par}{}{}
\makeatother
% Allow footnotes in longtable head/foot
\IfFileExists{footnotehyper.sty}{\usepackage{footnotehyper}}{\usepackage{footnote}}
\makesavenoteenv{longtable}
\usepackage{graphicx}
\makeatletter
\def\maxwidth{\ifdim\Gin@nat@width>\linewidth\linewidth\else\Gin@nat@width\fi}
\def\maxheight{\ifdim\Gin@nat@height>\textheight\textheight\else\Gin@nat@height\fi}
\makeatother
% Scale images if necessary, so that they will not overflow the page
% margins by default, and it is still possible to overwrite the defaults
% using explicit options in \includegraphics[width, height, ...]{}
\setkeys{Gin}{width=\maxwidth,height=\maxheight,keepaspectratio}
% Set default figure placement to htbp
\makeatletter
\def\fps@figure{htbp}
\makeatother
\setlength{\emergencystretch}{3em} % prevent overfull lines
\providecommand{\tightlist}{%
  \setlength{\itemsep}{0pt}\setlength{\parskip}{0pt}}
\setcounter{secnumdepth}{-\maxdimen} % remove section numbering
\ifLuaTeX
  \usepackage{selnolig}  % disable illegal ligatures
\fi
\IfFileExists{bookmark.sty}{\usepackage{bookmark}}{\usepackage{hyperref}}
\IfFileExists{xurl.sty}{\usepackage{xurl}}{} % add URL line breaks if available
\urlstyle{same} % disable monospaced font for URLs
\hypersetup{
  pdftitle={Data Structures},
  pdfauthor={Alex Takoudes},
  hidelinks,
  pdfcreator={LaTeX via pandoc}}

\title{Data Structures}
\author{Alex Takoudes}
\date{2023-01-17}

\begin{document}
\maketitle

\hypertarget{reviewing-assignment-3}{%
\subsubsection{Reviewing Assignment 3}\label{reviewing-assignment-3}}

\begin{Shaded}
\begin{Highlighting}[]
\NormalTok{queue }\OtherTok{\textless{}{-}} \FunctionTok{c}\NormalTok{(}\StringTok{"sheep"}\NormalTok{, }\StringTok{"fox"}\NormalTok{, }\StringTok{"owl"}\NormalTok{, }\StringTok{"ant"}\NormalTok{)}

\CommentTok{\#you can modify a specific index in the vector using []}
\NormalTok{queue[}\DecValTok{5}\NormalTok{] }\OtherTok{\textless{}{-}} \StringTok{"serpent"}
\NormalTok{queue}
\end{Highlighting}
\end{Shaded}

\begin{verbatim}
## [1] "sheep"   "fox"     "owl"     "ant"     "serpent"
\end{verbatim}

\begin{Shaded}
\begin{Highlighting}[]
\CommentTok{\#you can remove an index by using the "{-}"}
\NormalTok{queue }\OtherTok{\textless{}{-}}\NormalTok{ queue[}\SpecialCharTok{{-}}\DecValTok{5}\NormalTok{]}
\NormalTok{queue}
\end{Highlighting}
\end{Shaded}

\begin{verbatim}
## [1] "sheep" "fox"   "owl"   "ant"
\end{verbatim}

\begin{center}\rule{0.5\linewidth}{0.5pt}\end{center}

\hypertarget{data-structures}{%
\subsection{Data Structures}\label{data-structures}}

\begin{longtable}[]{@{}lll@{}}
\toprule()
Dimensions & Homogeneous & Heterogeneous \\
\midrule()
\endhead
\textbf{1-Dimension} & Atomic Vector & List \\
\textbf{2-Dimension} & Matrix & Data Frame \\
\bottomrule()
\end{longtable}

\hypertarget{vector-properties}{%
\subsubsection{Vector Properties}\label{vector-properties}}

\begin{itemize}
\tightlist
\item
  All atomic vectors are the same data type
\item
  When you use \texttt{c()} to assemble different types, R
  \textbf{coerces} them
\item
  c prioritizes different data types

  \begin{enumerate}
  \def\labelenumi{\arabic{enumi}.}
  \tightlist
  \item
    Logical (Boolean)
  \item
    Integer
  \item
    Double (floats)
  \item
    Character (strings)
  \end{enumerate}
\end{itemize}

\begin{Shaded}
\begin{Highlighting}[]
\CommentTok{\#Doubles (floats)}
\NormalTok{a }\OtherTok{\textless{}{-}} \FunctionTok{c}\NormalTok{(}\DecValTok{2}\NormalTok{, }\FloatTok{2.0}\NormalTok{)}
\FunctionTok{typeof}\NormalTok{(a)}
\end{Highlighting}
\end{Shaded}

\begin{verbatim}
## [1] "double"
\end{verbatim}

\begin{Shaded}
\begin{Highlighting}[]
\CommentTok{\#}
\NormalTok{b }\OtherTok{\textless{}{-}} \FunctionTok{c}\NormalTok{(}\StringTok{"purple"}\NormalTok{, }\StringTok{"green"}\NormalTok{)}
\NormalTok{d }\OtherTok{\textless{}{-}} \FunctionTok{c}\NormalTok{(a, b)}
\FunctionTok{typeof}\NormalTok{(d)}
\end{Highlighting}
\end{Shaded}

\begin{verbatim}
## [1] "character"
\end{verbatim}

\hypertarget{comparison-operators-yield-a-logical-result}{%
\paragraph{Comparison operators yield a logical
result}\label{comparison-operators-yield-a-logical-result}}

\begin{Shaded}
\begin{Highlighting}[]
\NormalTok{a }\OtherTok{\textless{}{-}} \FunctionTok{runif}\NormalTok{(}\DecValTok{10}\NormalTok{)}
\NormalTok{a }\SpecialCharTok{\textgreater{}} \FloatTok{0.5} \CommentTok{\#conditional statement}
\end{Highlighting}
\end{Shaded}

\begin{verbatim}
##  [1]  TRUE  TRUE  TRUE FALSE FALSE FALSE  TRUE  TRUE FALSE FALSE
\end{verbatim}

\begin{Shaded}
\begin{Highlighting}[]
\CommentTok{\# How many elements in the vector are \textgreater{} 0.5}
\FunctionTok{sum}\NormalTok{(a }\SpecialCharTok{\textgreater{}} \FloatTok{0.5}\NormalTok{)}
\end{Highlighting}
\end{Shaded}

\begin{verbatim}
## [1] 5
\end{verbatim}

\hypertarget{vectorization}{%
\paragraph{Vectorization}\label{vectorization}}

\begin{itemize}
\tightlist
\item
  Add a constant to a vector
\end{itemize}

\begin{Shaded}
\begin{Highlighting}[]
\NormalTok{z }\OtherTok{\textless{}{-}} \FunctionTok{c}\NormalTok{(}\DecValTok{10}\NormalTok{, }\DecValTok{20}\NormalTok{, }\DecValTok{30}\NormalTok{)}
\NormalTok{z}
\end{Highlighting}
\end{Shaded}

\begin{verbatim}
## [1] 10 20 30
\end{verbatim}

\begin{Shaded}
\begin{Highlighting}[]
\NormalTok{z }\SpecialCharTok{+} \DecValTok{1} \CommentTok{\#performs element{-}wise addition of 1}
\end{Highlighting}
\end{Shaded}

\begin{verbatim}
## [1] 11 21 31
\end{verbatim}

\begin{Shaded}
\begin{Highlighting}[]
\NormalTok{y }\OtherTok{\textless{}{-}} \FunctionTok{c}\NormalTok{(}\DecValTok{1}\NormalTok{, }\DecValTok{2}\NormalTok{, }\DecValTok{3}\NormalTok{)}
\NormalTok{z }\SpecialCharTok{+}\NormalTok{ y }\CommentTok{\#results in an element by element operation on the vector}
\end{Highlighting}
\end{Shaded}

\begin{verbatim}
## [1] 11 22 33
\end{verbatim}

\hypertarget{recycling}{%
\paragraph{Recycling}\label{recycling}}

\begin{itemize}
\tightlist
\item
  If vector lengths aren't equal
\end{itemize}

\begin{Shaded}
\begin{Highlighting}[]
\NormalTok{z }\OtherTok{\textless{}{-}} \FunctionTok{c}\NormalTok{(}\DecValTok{10}\NormalTok{, }\DecValTok{20}\NormalTok{, }\DecValTok{30}\NormalTok{)}
\NormalTok{x }\OtherTok{\textless{}{-}} \FunctionTok{c}\NormalTok{(}\DecValTok{1}\NormalTok{, }\DecValTok{2}\NormalTok{)}
\NormalTok{z }\SpecialCharTok{+}\NormalTok{ x}
\end{Highlighting}
\end{Shaded}

\begin{verbatim}
## Warning in z + x: longer object length is not a multiple of shorter object
## length
\end{verbatim}

\begin{verbatim}
## [1] 11 22 31
\end{verbatim}

\hypertarget{simulating-data}{%
\paragraph{Simulating Data}\label{simulating-data}}

\begin{itemize}
\tightlist
\item
  runif (Flat distribution)
\item
  rnorm (Curved distribution)
\end{itemize}

\begin{Shaded}
\begin{Highlighting}[]
\FunctionTok{set.seed}\NormalTok{(}\DecValTok{123}\NormalTok{) }\CommentTok{\#first set will always be the same numbers with seed}
\NormalTok{unif }\OtherTok{\textless{}{-}} \FunctionTok{runif}\NormalTok{(}\AttributeTok{n=}\DecValTok{100}\NormalTok{, }\AttributeTok{min=}\DecValTok{5}\NormalTok{, }\AttributeTok{max=}\DecValTok{10}\NormalTok{)}
\FunctionTok{hist}\NormalTok{(unif)}
\end{Highlighting}
\end{Shaded}

\includegraphics{DataStructures_files/figure-latex/unnamed-chunk-6-1.pdf}

\begin{Shaded}
\begin{Highlighting}[]
\NormalTok{rNN }\OtherTok{\textless{}{-}} \FunctionTok{rnorm}\NormalTok{(}\DecValTok{100}\NormalTok{) }\CommentTok{\#mean = 0, st.dev = 1}
\FunctionTok{mean}\NormalTok{(rNN) }\CommentTok{\#should be close to 0}
\end{Highlighting}
\end{Shaded}

\begin{verbatim}
## [1] -0.05374608
\end{verbatim}

\begin{Shaded}
\begin{Highlighting}[]
\FunctionTok{hist}\NormalTok{(rNN)}
\end{Highlighting}
\end{Shaded}

\includegraphics{DataStructures_files/figure-latex/unnamed-chunk-6-2.pdf}

\begin{center}\rule{0.5\linewidth}{0.5pt}\end{center}

\hypertarget{matrix}{%
\subsection{Matrix}\label{matrix}}

\begin{itemize}
\tightlist
\item
  2 dimensional (rows and columns)
\item
  Homogeneous data type
\item
  Is an atomic vector organized into rows and columns
\end{itemize}

\begin{Shaded}
\begin{Highlighting}[]
\NormalTok{my\_vec }\OtherTok{\textless{}{-}} \DecValTok{1}\SpecialCharTok{:}\DecValTok{12}
\NormalTok{m }\OtherTok{\textless{}{-}} \FunctionTok{matrix}\NormalTok{(}\AttributeTok{data =}\NormalTok{ my\_vec, }\AttributeTok{nrow =} \DecValTok{4}\NormalTok{) }\CommentTok{\#builds it by columns by default}
\NormalTok{m}
\end{Highlighting}
\end{Shaded}

\begin{verbatim}
##      [,1] [,2] [,3]
## [1,]    1    5    9
## [2,]    2    6   10
## [3,]    3    7   11
## [4,]    4    8   12
\end{verbatim}

\begin{Shaded}
\begin{Highlighting}[]
\NormalTok{m }\OtherTok{\textless{}{-}} \FunctionTok{matrix}\NormalTok{(}\AttributeTok{data =}\NormalTok{ my\_vec, }\AttributeTok{ncol =} \DecValTok{3}\NormalTok{, }\AttributeTok{byrow =}\NormalTok{ T)}
\NormalTok{m}
\end{Highlighting}
\end{Shaded}

\begin{verbatim}
##      [,1] [,2] [,3]
## [1,]    1    2    3
## [2,]    4    5    6
## [3,]    7    8    9
## [4,]   10   11   12
\end{verbatim}

\begin{center}\rule{0.5\linewidth}{0.5pt}\end{center}

\hypertarget{lists}{%
\subsection{Lists}\label{lists}}

\begin{itemize}
\tightlist
\item
  1 dimensional (same as vectors)
\item
  Can hold different data types
\end{itemize}

\begin{Shaded}
\begin{Highlighting}[]
\NormalTok{myList }\OtherTok{\textless{}{-}} \FunctionTok{list}\NormalTok{(}\DecValTok{1}\SpecialCharTok{:}\DecValTok{10}\NormalTok{, }\FunctionTok{matrix}\NormalTok{(}\DecValTok{1}\SpecialCharTok{:}\DecValTok{8}\NormalTok{, }\AttributeTok{nrow=}\DecValTok{4}\NormalTok{, }\AttributeTok{byrow=}\NormalTok{T), letters[}\DecValTok{1}\SpecialCharTok{:}\DecValTok{3}\NormalTok{], pi)}
\FunctionTok{str}\NormalTok{(myList) }\CommentTok{\#structure of list}
\end{Highlighting}
\end{Shaded}

\begin{verbatim}
## List of 4
##  $ : int [1:10] 1 2 3 4 5 6 7 8 9 10
##  $ : int [1:4, 1:2] 1 3 5 7 2 4 6 8
##  $ : chr [1:3] "a" "b" "c"
##  $ : num 3.14
\end{verbatim}

\hypertarget{subsetting-lists}{%
\paragraph{Subsetting lists}\label{subsetting-lists}}

\begin{itemize}
\tightlist
\item
  Using \texttt{{[}{]}} gives you a single item BUT not the elements
\end{itemize}

\begin{Shaded}
\begin{Highlighting}[]
\NormalTok{myList }\OtherTok{\textless{}{-}} \FunctionTok{list}\NormalTok{(}\DecValTok{1}\SpecialCharTok{:}\DecValTok{10}\NormalTok{, }\FunctionTok{matrix}\NormalTok{(}\DecValTok{1}\SpecialCharTok{:}\DecValTok{8}\NormalTok{, }\AttributeTok{nrow=}\DecValTok{4}\NormalTok{, }\AttributeTok{byrow=}\NormalTok{T), letters[}\DecValTok{1}\SpecialCharTok{:}\DecValTok{3}\NormalTok{], pi)}
\NormalTok{myList[}\DecValTok{4}\NormalTok{]}
\end{Highlighting}
\end{Shaded}

\begin{verbatim}
## [[1]]
## [1] 3.141593
\end{verbatim}

\begin{Shaded}
\begin{Highlighting}[]
\CommentTok{\#myList[4] {-} 3 returns an error because it\textquotesingle{}s still compartamentalized}
\FunctionTok{typeof}\NormalTok{(myList[}\DecValTok{4}\NormalTok{])}
\end{Highlighting}
\end{Shaded}

\begin{verbatim}
## [1] "list"
\end{verbatim}

\begin{Shaded}
\begin{Highlighting}[]
\CommentTok{\#to grab object, use [[]]}
\NormalTok{myList[[}\DecValTok{4}\NormalTok{]] }\SpecialCharTok{{-}} \DecValTok{3}
\end{Highlighting}
\end{Shaded}

\begin{verbatim}
## [1] 0.1415927
\end{verbatim}

\begin{Shaded}
\begin{Highlighting}[]
\NormalTok{myList[[}\DecValTok{2}\NormalTok{]][}\DecValTok{4}\NormalTok{,}\DecValTok{1}\NormalTok{] }\CommentTok{\#access the matrix, get the 4th row, 1st column}
\end{Highlighting}
\end{Shaded}

\begin{verbatim}
## [1] 7
\end{verbatim}

\begin{Shaded}
\begin{Highlighting}[]
\FunctionTok{c}\NormalTok{(myList[[}\DecValTok{1}\NormalTok{]], myList[[}\DecValTok{2}\NormalTok{]]) }\CommentTok{\#to obtain multiple elements within list}
\end{Highlighting}
\end{Shaded}

\begin{verbatim}
##  [1]  1  2  3  4  5  6  7  8  9 10  1  3  5  7  2  4  6  8
\end{verbatim}

\hypertarget{name-list-items-when-they-are-created}{%
\paragraph{Name list items when they are
created}\label{name-list-items-when-they-are-created}}

\begin{Shaded}
\begin{Highlighting}[]
\NormalTok{myList2 }\OtherTok{\textless{}{-}} \FunctionTok{list}\NormalTok{(}\AttributeTok{Tester =}\NormalTok{ F, }\AttributeTok{littleM =} \FunctionTok{matrix}\NormalTok{(}\DecValTok{1}\SpecialCharTok{:}\DecValTok{9}\NormalTok{, }\AttributeTok{nrow=}\DecValTok{3}\NormalTok{))}
\NormalTok{myList2}\SpecialCharTok{$}\NormalTok{Tester }\CommentTok{\#calls the element by it\textquotesingle{}s name}
\end{Highlighting}
\end{Shaded}

\begin{verbatim}
## [1] FALSE
\end{verbatim}

\begin{Shaded}
\begin{Highlighting}[]
\NormalTok{myList2}\SpecialCharTok{$}\NormalTok{littleM[}\DecValTok{2}\NormalTok{,}\DecValTok{3}\NormalTok{] }\CommentTok{\#access the matrix and pull the 2nd row, 3rd column}
\end{Highlighting}
\end{Shaded}

\begin{verbatim}
## [1] 8
\end{verbatim}

\begin{Shaded}
\begin{Highlighting}[]
\NormalTok{myList2}\SpecialCharTok{$}\NormalTok{littleM[}\DecValTok{2}\NormalTok{,]  }\CommentTok{\#access the matrix and pull the entire 2nd row}
\end{Highlighting}
\end{Shaded}

\begin{verbatim}
## [1] 2 5 8
\end{verbatim}

\hypertarget{unlist-to-string-everything-back-to-vector}{%
\paragraph{Unlist to string everything back to
vector}\label{unlist-to-string-everything-back-to-vector}}

\begin{Shaded}
\begin{Highlighting}[]
\NormalTok{myList2 }\OtherTok{\textless{}{-}} \FunctionTok{list}\NormalTok{(}\AttributeTok{Tester =}\NormalTok{ F, }\AttributeTok{littleM =} \FunctionTok{matrix}\NormalTok{(}\DecValTok{1}\SpecialCharTok{:}\DecValTok{9}\NormalTok{, }\AttributeTok{nrow=}\DecValTok{3}\NormalTok{))}
\NormalTok{unrolled }\OtherTok{\textless{}{-}} \FunctionTok{unlist}\NormalTok{(myList2)}
\NormalTok{unrolled}
\end{Highlighting}
\end{Shaded}

\begin{verbatim}
##   Tester littleM1 littleM2 littleM3 littleM4 littleM5 littleM6 littleM7 
##        0        1        2        3        4        5        6        7 
## littleM8 littleM9 
##        8        9
\end{verbatim}

\end{document}
